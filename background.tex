\documentclass[Main]{subfiles}
\begin{document}
% For solving a problem in the hospital domain Partial Order Planning (POP) and Hierarchical Task Network (HTN) is used.
% At the time of the competition POP used Multi Queue Best First Search \cite{hector2013a} (MQBFS) is used in a naive implementation. \par
% The MQBFS is a way to use multiple queues to explore states, normally these queues are divided into a helpful and a non-helpful list. 
% The algorithm then alternates between the two queues, using the node with the lowest heuristics.
% This also means that the heuristic is calculated when a node is expanded, which results in a smaller overhead of heuristics. \par
% Another thing used is a global goal order constraint, where the order of the goals is split into two.
% \begin{enumerate}
% 	\item Is the goal blocking for another goal, using the initial state as offset for calculation
% 	\item The length of the solution of each goal
% \end{enumerate}
% This allows for goals to be ordered before it is split into partial plans using POP.


% - What is POP (Tuple notation)
We define a plan for solving a level in the Block World domain, by a tuple of four elements $(A,O,L,P)$ \citep{Weld1994}\citep{Russell2003}, where $A$ is a list of actions, $O$ is a set of ordering constraints, $L$ is a set of weak links between actions, and $P$ is a set of open preconditions. 
We've followed the definition given in \cite{Russell2003}, with some inspiration from \cite{Weld1994}.
Though we do use a lot of key-concepts from Partial Order (Least Commitment) Planning, we do \textit{not} follow the specification rigorously in our implementation.

% - MQBFS
We've also drawn some inspiration from Multi Queue Best First Search (MQBFS) \citep[p.~38]{hector2013a}, where we alternate between two frontier queues (i.e. queues containing un-expanded nodes).


% - Bidding / MA
% - Plan merging

\end{document}
