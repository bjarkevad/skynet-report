\documentclass[Main]{subfiles}
\begin{document}

% In this report we've presented a solution, that \todo[inline]{Say something.....}
Our planner has been able to create good plans for many levels, though we're aware of the shortcomings of the implementation.

% \todo[inline]{\textbf{Application awareness?}}

% - Speed over optimal solution ?

% The two main points to be taken from this paper are as follows: \todo[inline]{PHRASING!}

When implementing our Least Commitment Planner, we reached the somewhat common issue of performance versus quality of the output; more specifically:

\begin{enumerate}
	\item Processing time, i.e. how much CPU time is needed to create a plan
	\item Optimal solution, i.e. the number of actions taken to solve a level
\end{enumerate}

These two factors often present a trade off, in the sense that it's hard to improve one, without negatively affecting the other. 
For example would trying to get a more optimal solution, typically cause more of the state space to be searched, which has a negative impact on running time.
At the same time would trying to quickly find a solution mean that some shortcuts are necessary, which in turn has a negative impact on the quality of the solution.

The choice of data structures also had, as expected, a very drastic effect on performance ---
this is especially evident in the difference in performance characteristics when using either BFS or MQBFS, as seen in \autoref{sec:experiments}.

A somewhat heavier use of pre-processing was also considered when working with Hierarchical Task Networks, but was scraped in favor of our Partial Order Planner. 
POP, which had shown some rather impressive results, both by being able to solve more level, but also by doing so quicker, then became the main focus point of the paper.

\end{document}
