\textbf{Benchmark - SAMAStersF13.lvl}


The most interesting experiments conducted with the used search strategy involved two strategies. Best-first search using A* evaluation and a modified version, multi-queue best-first search using A* evaluation. The strategy applies to both relaxed and full problem search. 
The multi-queue best-first search utilizes two queues, one active ``main queue'' and one ``back-up queue''. The strategy is a very naive implementation of a multi-queue BFS, as a normal implementation alternates between expansions of the nodes in both queues \cite{hector2013a}, whereas our implementation relies heavily on the ``main queue'' being the \textif{``correct queue''}. Every time a node is expanded, all other nodes in the ``main queue'' are added to the ``back-up queue'' and removed from the main, such that the child nodes from the expanded node are the only nodes in the ``main queue''. 



20 second run 

BFS:

[Client said] Search starting with strategy Best-first Search (PriorityQueue) using A* evaluation
[Client said] #Explored:    0, #Frontier:   1, Time: 0.00 s     [Used: 291.41 MB, Free: 316.59 MB, Alloc: 608.00 MB, MaxAlloc: 7282.00 MB]

[Client said] #Explored: 79000, #Frontier: 11774, Time: 20.33 s     [Used: 893.79 MB, Free: 727.21 MB, Alloc: 1621.00 MB, MaxAlloc: 7282.00 MB]


Multi queue BFS:

[Client said] Search starting with strategy Multi-Best-first Search (PriorityQueue) using A* evaluation
[Client said] #Explored:    0, #Frontier:   1, Time: 0.00 s     [Used: 260.84 MB, Free: 219.66 MB, Alloc: 480.50 MB, MaxAlloc: 7282.00 MB]

[Client said] #Explored: 256978, #Frontier: 198642, Time: 20.47 s   [Used: 4007.45 MB, Free: 1369.05 MB, Alloc: 5376.50 MB, MaxAlloc: 7282.00 MB]


\textbf{Analysis}
From the results it is very clear that the multi queue BFS strategy explores more states than BFS. However, the memory consumption is also a great deal higher. Of course this is mainly due to the size of the frontiers combined, which counts over 10 times more than the BFS strategy. 


